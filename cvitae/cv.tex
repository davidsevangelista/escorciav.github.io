% LaTeX file for resume 
% This file uses the resume document class (res.cls)

\documentclass[margin]{res} 
% the margin option causes section titles to appear to the left of body text 
\textwidth=5.2in % increase textwidth to get smaller right margin
%\usepackage{helvetica} % uses helvetica postscript font (download helvetica.sty)
%\usepackage{newcent}   % uses new century schoolbook postscript font 

\begin{document} 
 
\name{Victor A. Escorcia\\[12pt]} % the \\[12pt] adds a blank line after name
 
\address{{\bf Present Address} \\ Al Khawarizmi Building (Bldg 1), 2106-WS06
          \\ Thuwal 23955-6900, Kingdom of Saudi Arabia}
\address{{\bf Contact} \\ escorciav@gmail.com \\ +966 56 948 5932}
 
\begin{resume} 
 
\section{Research} 
Computer Vision, Machine Learning, Data Science.


\section{Education}
\textbf{Universidad del Norte} \textit{(Barranquilla, Colombia)}, February 2014 \\
M.Sc. in Electronic Engineering\\
\bigskip Adviser: Professor Juan Carlos Niebles\\
\textbf{Universidad del Norte} \textit{(Barranquilla, Colombia)}, September 2012 \\
B.Eng., Electrical Engineering\\
\textbf{Universidad del Norte} \textit{(Barranquilla, Colombia)}, February 2012 \\
B.Eng., Electronic Engineering\\
 

\section{Experience}
 \textit{Visiting Student}\\
 Visual Computing Center, King Abdullah University of Science and Technology.
 Saudi Arabia. \hfill March 2014 - Present \\
 \begin{itemize} \itemsep -2pt %reduce space between items 
 \item Developing algorithms for performing human action recognition and human
       tracking using multiple cameras.
 \item Perform research activities e.g. analyze recent literature in computer
       vision, equipment selection, leading reading sessions, etc.
 \end{itemize}
 
 \textit{Research Assistant}\\
 Computer Vision Lab, Universidad del Norte, Colombia. \hfill February 2012 - 2014 \\
 \begin{itemize} \itemsep -2pt %reduce space between items 
  \item Developing algorithms for performing human action recognition using
 human-object interactions as cues.
 \item Perform research activities e.g. analyze recent literature in computer
       vision, equipment selection, leading reading sessions, etc.
 \end{itemize}
 
 \textit{Peer Tutor}\\
 CREE, Universidad del Norte, Colombia \hfill January  2009 - 2011 \\
 \begin{itemize} \itemsep -2pt %reduce space between items
  \item Led review sessions for MAT1111: Single Variable Calculus, FIS1033:
 Physics II: Electricity and Magnetism, IEN4030: Analog Electronics,
 IEN7140: Power Electronics.
  \end{itemize}

%{\bf System Consultant,} Fleet Van Lines, Bayridge, NY \hfill  Summer 1984
%\begin{itemize} \itemsep -2pt %reduce space between items
%\item Researched, implemented new computer accounting 
%                 system 
%\item Customized existing software for inventory 
%                 management 
%\item Trained employees on both accounting and inventory 
%                 systems 
%\end{itemize}


\section{Publications} 
\begin{itemize} \itemsep -2pt %reduce space between items
\item Victor Escorcia, Juan Carlos Niebles. \textbf{Spatio-Temporal 
Human-Object Interactions for Action Recognition in Videos}. International
Conference on Computer Vision, 1st Workshop on Understanding Human Activities:
Context and Interactions. Sydney, Australia, December 2013.
\smallskip

\textit{Abstract:} We introduce a new method for representing the dynamics of
human-object interactions in videos. Previous algorithms tend to focus on
modeling the spatial relationships between objects and actors, but ignore the
evolving nature of this relationship through time. Our algorithm captures the
dynamic nature of human-object interactions by modeling how these patterns
evolve with respect to time. Our experiments show that encoding such temporal
evolution is crucial for correctly discriminating human actions that involve
similar objects and spatial human-object relationships, but only differ on the
temporal aspect of the interaction, e.g. answer phone and dial phone.
We validate our approach on two human activity datasets and show performance
improvements over competing state-of-the-art representations.
\medskip

\item Victor Escorcia, Maria Davila, Mani Golparvar-Fard and Juan Carlos
Niebles. \textbf{Automated Vision-based Recognition of Construction Worker
Actions for Building Interior Construction Operations Using RGBD Cameras}.
Construction Research Congress 2012. West Lafayette, Indiana, May 2012.
\smallskip

\textit{Abstract:} In this paper we present a novel method for reliable
recognition of construction workers and their actions using color and depth
data from a Microsoft Kinect sensor. Our algorithm is based on machine learning
techniques, in which meaningful visual features are extracted based on the
estimated body pose of workers. We adopt a bag-of-poses representation for
worker actions and combine it with powerful discriminative classifiers to
achieve accurate action recognition. The discriminative framework is able to
focus on the visual aspects that are distinctive and can detect and recognize
actions from different workers. We train and test our algorithm by using 80
videos from four workers involved in five drywall related construction
activities. These videos were all collected from drywall construction
activities inside of an under construction dining hall facility. The proposed
algorithm is further validated by recognizing the actions of a construction
worker that was never seen before in the training dataset. Experimental
results show that our method achieves an average precision of 85.28 percent.
The results reflect the promise of the proposed method for automated assessment
of craftsmen productivity, safety, and occupational health at indoor
environments.

\end{itemize}
               

\section{Honors and Awards}
Scientific Diploma to the master thesis graded as Cum Laude.
Universidad del Norte. Barranquilla, Colombia. February 2014.

Joven Investigador Scholarship to support young researchers.
Colciencias \& Universidad del Norte. Barranquilla, Colombia. February 2013.

Silver Medal to the Graduate of Excellence of Electrical Engineering.
GPA 4.55/5.0, ranked ${1^{{\rm{st}}}}$ in graduating class over 15.
Universidad del Norte. Barranquilla, Colombia. September 2012.

Silver Medal to the Graduate of Excellence of Electronic Engineering.
GPA 4.55/5.0, ranked ${1^{{\rm{st}}}}$ in graduating class over 40.
Universidad del Norte. Barranquilla, Colombia. February 2012.

Dean's List. Universidad del Norte. Barranquilla, Colombia. 2007-2011.

Roble Amarillo Scholarship. Universidad del Norte. 
Barranquilla, Colombia. 2007-2011.

Outstanding High School Graduate. Instituto La-Salle, Alumni Association.
Barranquilla, Colombia. 2006.


% Tabulate Computer Skills; p{3in} defines paragraph 3 inches wide
\section{Computer \\ Skills}
   \begin{tabular}{l p{3in}}
    \underline{Languages}: & Matlab, Python (Intermediate), C++ (Basic). \\

    \underline{Software}: & Git, LaTeX, Microsoft Office, AutoCAD, Illustrator (Basic). \\
    
    \underline{Operative Systems}: & Windows, Linux. \\
 \end{tabular}

\end{resume} 
\end{document} 

